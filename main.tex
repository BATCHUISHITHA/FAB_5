\documentclass{article}
\usepackage{graphicx}
\usepackage{float}
\usepackage{multirow}
\usepackage{listings}
\begin{document}
\begin{figure}[H]
\centering
\includegraphics[scale = 0.2 ]{./figs/logo.png}
\end{figure}

\begin{center}
	\textbf{\LARGE Indian Institute of Technology Hyderabad} \\
	\vspace{1cm}
	\Large Digital Fabrication, 2023-24\\
	\vspace{0.3cm}
	\large Report  On\\

	\rule{\linewidth}{0.5pt} \\
	\vspace{0.2cm}
	\textbf{\LARGE 3D Printing of  \\ \vspace{0.3cm} Drone} \\
	\vspace{0.1cm}
	\rule{\linewidth}{0.5pt} \\

	\vspace{1.5cm}
        \textbf{Project by: FAB 5}\\
	\quad EE23BTech11016 - Aditi Dure, \\
	\quad EE23BTech11011 - Batchu Ishitha, \\
	\quad EE23BTech11008 - Siva Meenakshi, \\
	\quad EE23BTech11018 - Mohana Eppala, \\
	\quad EE23BTech11001 - Aashna Sahu \\
	
	\textbf{Instructors: }\\
	\quad Prof. Anil Agrawal \\
	
	

	\vspace{1cm}
	\date{}
\end{center}

%\maketitle

\section{Selection of Cad Model}
\subsection{Object Description}
Drones are unmanned aerial vehicles (UAVs) that can be remotely piloted or operate autonomously using software-controlled flight plans. They vary widely in size, from small hobbyist drones to large military or commercial drones. They often have cameras for surveillance, delivery capabilities, or recreational use like aerial photography and videography.\\The body of a drone, also known as the frame or chassis, is the central structure that holds all the components together.
Propellers are the rotating blades on a drone that generate lift and propulsion.
Legs or Landing gear are the components that support the drone when it's on the ground or during takeoff and landing. 
\subsection{Special Features}
The special features of our model are distinctive body shape and movable propellers. 
\subsection{Why was "Drone" selected as Project? }
Drone is selected as our project due to the following reasons:
\begin{enumerate}
\item Drones offer a complex and challenging design. They involve various components such as the frame, propulsion system,  and moving parts like propellers. Designing and 3D printing a drone allows for a comprehensive exploration of these complexities.
\item Drones integrate multiple technologies such as aerodynamics, electronics, software, and mechanics. Designing a drone model involves understanding and incorporating these technologies, providing a holistic learning experience.
\item  Drones are functional objects with specific purposes such as aerial photography, surveillance, delivery, or recreational flying.
\item Drones are associated with innovation, cutting-edge technology.
\item Drones are educational tools that can be used to teach principles of physics, engineering, robotics, programming, and more. Designing and 3D printing a drone model facilitates hands-on learning and skill development in these areas.
\end{enumerate}
\section{Making of Drone on Solid Edge}
We have divided the drone to four parts as discussed in object description, which are, 
\begin{enumerate}
\item Main Body 
\item Projections
\item Propeller
\item Screw
\end{enumerate}
\subsection{Parts}
\subsubsection{Body}
\begin{figure}[H]
\centering
\includegraphics[scale = 0.3 ]{./figs/body1.jpeg}
\end{figure}
This part has two parts which are mirror images of each other and are attached to each other externally. It is done because we needed a gap in this in order to attach a screw thriugh which the projections are attached. \\
The tools used in this are:
\begin{itemize}
\item \textbf{Extrusion: }It is used almost everywhere to convert 2D to 3D.
\item \textbf{Round: } Used in order to avoid sharp edges.
\item \textbf{Mirror: } Used because the body was symmetric as well and to get another symmetric body to attach from below.
\end{itemize}
\begin{figure}[H]
\centering
\includegraphics[scale = 0.3 ]{./figs/body2.jpeg}
\end{figure}
\subsection{Projections}
\subsection{Propeller}
\subsection{Screws}

\section{Conversion to STL file}
\begin{figure}[H]
\centering
\includegraphics[scale = 0.3 ]{./figs/1.jpeg}
\end{figure}
\begin{figure}[H]
\centering
\includegraphics[scale = 0.3 ]{./figs/2.jpeg}
\end{figure}\begin{figure}[H]
\centering
\includegraphics[scale = 0.3 ]{./figs/3.jpeg}
\end{figure}
\begin{figure}[H]
\centering
\includegraphics[scale = 0.3 ]{./figs/4.jpeg}
\end{figure}
\begin{figure}[H]
\centering
\includegraphics[scale = 0.3 ]{./figs/5.jpeg}
\end{figure}
\begin{figure}[H]
\centering
\includegraphics[scale = 0.3 ]{./figs/6.jpeg}
\end{figure}
\begin{figure}[H]
\centering
\includegraphics[scale = 0.3 ]{./figs/7.jpeg}
\end{figure}
\begin{figure}[H]
\centering
\includegraphics[scale = 0.3 ]{./figs/8.jpeg}
\end{figure}
\begin{figure}[H]
\centering
\includegraphics[scale = 0.3 ]{./figs/9.jpeg}
\end{figure}
\begin{figure}[H]
\centering
\includegraphics[scale = 0.3 ]{./figs/10.jpeg}
\end{figure}
\begin{figure}[H]
\centering
\includegraphics[scale = 0.3 ]{./figs/11.jpeg}
\end{figure}
\begin{figure}[H]
\centering
\includegraphics[scale = 0.3 ]{./figs/12.jpeg}
\end{figure}
\begin{figure}[H]
\centering
\includegraphics[scale = 0.3 ]{./figs/13.jpeg}
\end{figure}
\section{Final result as 3D Printing}
\subsection{Viewing Features}
\subsection{inaccuracies}


\section{Observations}
\subsection{Experience}
\subsection{Challenges}
\subsection{Learnings from 3D printing}




\end{document}
